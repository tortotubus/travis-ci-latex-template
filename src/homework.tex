\documentclass{lapchomework}
\usepackage{lapchomework}

% Document information
\title{Homework for Math 275}
\date{Spring 2022}
\author{Conor E. Olive}

%%%%%%%%%%%%%%%%%%%%%%%%%%%%%%%%% DOCUMENT %%%%%%%%%%%%%%%%%%%%%%%%%%%%%%%%%%%%%

% Begin document
\begin{document}

%\section*{Homework Assignments}
%\begin{table}[h]
%\centering
%\begin{tabular}{|l|l|l|}
%\hline
%\textbf{Date}     & \textbf{Section} & \textbf{Problems}   \\
%\hline
%February 07, 2022 & 1.3              & 5, 7, 11, 13, 15    \\
%February 09, 2022 & 2.2              & 3, 5, 9, 10, 17     \\
%February 14, 2022 & 2.1              & 3c, 9, 11K, 21       \\
%February 16, 2022 & 2.2              & 27, 28              \\
%February 23, 2022 & 2.3              & 1, 12, 16a          \\
%\hline
%\end{tabular}
%\end{table}
%\pagebreak

\tableofcontents
\pagebreak

\mychapter{1}{Introduction}
\mysection{3}{Classification of Differential Equations}

\begin{instructions}
In each of Problems 5 through 10, verify that each given function is a
solution of the differential equation.
\end{instructions}

\begin{problems}

\problem[5] $y^{\prime\prime}-y=0$; $y_1(t)=e^t$, $y_2(t)=\cosh{t}$

\begin{solution}

\step First, let us check if $y_1$ is a valid solution to the differential 
equation by first finding the second derivative of $y_1$:

\step \begin{gather*}
y_1 = e^t \\
y_1^{\prime\prime}=e^t
\end{gather*}

\step Next, substitute $y_1$ and $y_1^{\prime\prime}$ for $y$ and 
$y^{\prime\prime}$ respectively:

\step \begin{gather*}
e^{t} - e^{t} = 0\\
\boxed{0 = 0}
\end{gather*}

\step So yes, the function $y_1$ is a valid solution to the differential 
equation. Next, let us check if $y_2$ is a valid solution to the differential 
equation by first finding the second derivative of $y_2$:

\step \begin{gather*}
y_2 = \cosh{(t)}\\
y_2^{\prime\prime}=\cosh{(t)}
\end{gather*}

\step Next, substitute $y_2$ and $y_2^{\prime\prime}$ for $y$ and 
$y^{\prime\prime}$ respectively:

\step \begin{gather*}
\cosh{(t)} - \cosh{(t)} = 0\\
\boxed{0 = 0}
\end{gather*}

\step So yes, the function $y_2$ is a valid solution to the differential 
equation.
\end{solution}

\problem[7] $ty^{\prime}-y=t^2$; $y=3t+t^2$
\begin{solution}

\step Let us check if $y$ is a valid solution to the differential
equation by taking the first derivative:

\step \begin{gather*}
y=3t+t^2\\
y^{\prime}=3+2t
\end{gather*}

\step Next, let us substitute:

\step \begin{gather*}
ty^{\prime}-y=t^2\\
t(2+2t) - (3t + t^2) = t^2\\
2t^2 - t^2 + 2t - 3t = t^2\\
\boxed{t^2 - t \neq t^2}
\end{gather*}

\step So no, the function $y$ is not a valid solution to the differential
equation.

\end{solution}

\end{problems}

\begin{instructions}
In each of Problems 11 through 13, determine the values of $r$ for which
the given differential equation has solutions of the form $y = e^{rt}$.
\end{instructions}

\begin{problems}

\problem[11] $y^{\prime}+2y=0$

\begin{solution}

\step First, take the first derivative of $y=e^{rt}$:

\step \begin{align*}
y & = e^{rt} \\
y^{\prime} & = re^{rt}
\end{align*}

\step Next, substitute and solve for $r$:

\step \begin{align*}
y^{\prime} + 2y & = 0 \\
re^{rt} + 2e^{rt} & = 0 \\
re^{rt} & = -2e^{rt} \\
\frac{1}{e^{rt}} \cdot re^{rt} & = -2e^{rt} \cdot \frac{1}{e^{rt}} \\
\Aboxed{r & = -2}
\end{align*}
\end{solution}

\problem[13] $y^{\prime\prime\prime}-3y^{\prime\prime}+2y^{\prime}=0$

\begin{solution}

\step First, take the first, second, and third derivatives of $y=e^{rt}$:

\step \begin{align*}
y & = e^{rt} \\
y^{\prime} & = re^{rt} \\
y^{\prime\prime} & = r^2e^{rt} \\
y^{\prime\prime\prime} & = r^3e^{rt}
\end{align*}

\step Next, substitute:

\step \begin{align*}
y^{\prime\prime\prime}-3y^{\prime\prime}+2y^{\prime} & = 0 \\
r^3e^{rt} - 3(r^2e^{rt}) + 2(re^{rt}) & = 0
\end{align*}

\step Here, we can see that one solution for $r$ is $0$:

\step \begin{align*}
r^3e^{rt} - 3(r^2e^{rt}) + 2(re^{rt}) & = 0 \\
0 - 3(0) + 2(0) & = 0 \\
0 & = 0
\end{align*}

\step Continue finding other solutions:

\step \begin{gather*}
\frac{1}{re^{rt}} \cdot  \left[r^3e^{rt} - 3(r^2e^{rt}) + 2(re^{rt}) \right] = 
0 \\
r^2 - 3r = -2 \\
r^2 - 3r + \frac{9}{4} = \frac{1}{4} \\
\left(r-\frac{3}{2}\right)^2 = \frac{1}{4} \\
r - \frac{3}{2} = \pm \frac{1}{2}
\end{gather*}

\step Continue solving for each sign of $1/2$:

\step \begin{gather*}
r = \frac{1}{2} + \frac{3}{2} \\
\boxed{r = 2} \\
\\
r = - \frac{1}{2} + \frac{3}{2} \\
\boxed{r = 1}
\end{gather*}

\step Now we have found all solutions for $r$:

\step \begin{align*}
\Aboxed{r & = \{0, 1, 2\}}
\end{align*}

\end{solution}

\end{problems}

\begin{instructions}
In each of Problems 14 and 15, determine the values of r for which the
given differential equation has solutions of the form $y = t^r$ for $t > 0$.
\end{instructions}

\begin{problems}

\problem[15] $t^2y^{\prime\prime}-4ty^{\prime}+4y=0$

\begin{solution}

\step First, let us find the first and second derivatives of $y = t^r$:

\step \begin{align*}
y & = t^r \\
y^{\prime} & = rt^{r-1} \\
y^{\prime\prime} & = r(r-1)t^{r-2}
\end{align*}

\step Next, substitute:

\step \begin{align*}
t^2r(r-1)t^{r-2} - 4trt^{r-1} + 4t^r=0 \\
(t^2)(r^2 - r)(t^{r-2}) - 4trt^{r-1} + 4t^r = 0 \\
(t^r)(r^2 - r) - 4trt^{r-1} + 4t^r = 0 \\
r^2t^r - rt^r - 4trt^{r-1} + 4t^r = 0 \\
r^2t^r - rt^r - 4rt^r + 4t^r = 0 \\
t^r(r^2 - r - 4r + 4) = 0 \\
\end{align*}

\step Since $t > 0$, the term $t^r$ will not yield a solution.

\step \begin{align*}
t^r(r^2 - 5r + 4) = 0 \\
t^r(r - 4)(r - 1) = 0 \\
\Aboxed{r = \{1, 4\}}
\end{align*}

\end{solution}

\end{problems}

\mychapter{2}{First-Order Differential Equations}
\mysection{1}{Linear Differential Equations; Method of Integrating Factors}

\begin{instructions}
In each of Problems 1 through 8:
\begin{subinstructions}
\subinstruction \sout{Draw a direction field for the given differential 
equation.}
\subinstruction \sout{Based on an inspection of the direction field, describe 
how solutions behave for large t.}
\subinstruction Find the general solution of the given differential equation,
and use it to determine how solutions behave as $t \to \infty$.
\end{subinstructions}
\end{instructions}

\begin{problems}

\item [3] \begin{subproblems}

\subproblem [c] $y^{\prime}+y=te^{-t}+1$

\begin{solution}

\step We can recognize the following functions in the first-order linear 
differential equation:

\step \begin{align*}
P(t) & =  1 \\
G(t) & = te^{-t} + 1 
\end{align*}

\step Using this, we can find our integrating factor:

\step \begin{align*}
u(t) & = e^{\int P(t) dt} \\
u(t) & = e^{\int 1 dt} \\
u(t) & = e^t
\end{align*}

\step Now use our integrating factor:

\step \begin{align*}
y^{\prime} + y & = te^{-t} + 1 \\
e^t \cdot y^{\prime} + e^t \cdot y & = \left(te^{-t} + 1\right) \cdot e^t \\
y^{\prime}e^t + ye^t & = t + e^t \\
\left(e^ty\right)^{\prime} & = t + e^t \\
\int \left(e^ty\right)^{\prime} dt & = \int (t + e^t) dt \\
e^ty & = \frac{t^2}{2}+e^t+c\\
\Aboxed{y & = \frac{t^2e^{-t}}{2} + 1 + ce^{-t}}
\end{align*}

\item Analyzing this solution to the differential equation, we can see that
as $t \to \infty$, $y \to 1$:

\item \begin{align*}
\Aboxed{\lim_{t \to \infty} \left( \frac{t^2}{2e^t} + \frac{c}{e^t} 
+ 1 \right)& = 1}
\end{align*}

\end{solution}

\end{subproblems}

\end{problems}

\begin{instructions}
In each of Problems 9 through 12, find the solution of the given initial
value problem.
\end{instructions}

\begin{problems}

\problem [9] $y^{\prime}-y=2te^{2t}$, $y(0)=1$

\begin{solution}

\step \begin{align*}
P(t) & = -1 \\
G(t) & = 2te^{2t} 
\end{align*}

\step Now find the integrating factor $u(t)$:

\step \begin{align*}
u(t) & = e^{\int P(t) dt} \\
u(t) & = e^{- \int dt} \\
u(t) & = e^{-t}
\end{align*}

\step Now use the integrating factor:

\step \begin{align*}
e^{-t} \cdot y^{\prime} - e^{-t} \cdot y & = 2te^{2t} \cdot e^{-t} \\
y^{\prime}e^{-t} - ye^{-t} & = 2te^t \\
\left(e^{-t}y\right)^{\prime} & = 2te^{t} \\
\int \left(e^{-t}y\right)^{\prime} dt & = \int 2te^{t} dt \\
e^{-t}y & = 2e^t(t-1)+c \\
e^t \cdot e^{-t}y & = \left(2e^t(t-1)+c\right) \cdot e^t \\
y & = 2e^{2t}(t-1)+ce^t \\
y & = 2te^{2t} - 2e^{2t} + ce^t
\end{align*}

\step Now use the initial value given:

\step \begin{align*}
1 & = 2e^{0}(0-1)+ce^0 \\
1 & = 2(1)(-1)+c(1) \\
1 & = -2 + c \\
c & = 3
\end{align*}

\step Now substitute $c$:

\step \begin{align*}
\Aboxed{y & = 2te^{2t} - 2e^{2t} + 3e^t}
\end{align*}

\end{solution}

\pagebreak

\problem [11] $y^{\prime}+\frac{2}{t}=\frac{\cos{t}}{t^2}$, $y(\pi)=0$, $t > 0$

\begin{solution}

\step \begin{align*}
P(t) & = \frac{2}{t} \\
G(t) & = \frac{\cos{t}}{t^2}
\end{align*}

\step Now find the integrating factor:

\step \begin{align*}
u(t) & = e^{\int P(t) dt} \\
u(t) & = e^{\int 2t^{-1} dt} \\
u(t) & = e^{2\ln{t}} \\
u(t) & = t^2
\end{align*}

\step Now use the integrating factor

\step \begin{align*}
y^{\prime}+\frac{2}{t} & = \frac{\cos{t}}{t^2} \\
t^2 \cdot y^{\prime} + t^2 \cdot \frac{2}{t} & = \frac{\cos{t}}{t^2} \cdot t^2 \\
\left(t^2y\right)^{\prime} & = \cos{t} \\
\int \left(t^2y\right)^{\prime} dt & = \int \cos{t} dt \\
t^2y & = \sin{t} + c \\
y & =t^{-2}\sin{t} + ct^{-2}
\end{align*}

\step Now use the initial value to find $c$:

\step \begin{align*}
0 & = \pi^{-2}\sin{pi} + c\pi^{-2} \\
0 & = \sin{pi} + c \\
c & = -\sin{pi} \\
c & = 0
\end{align*}

\step Now use the value found for $c$:

\step \begin{align*}
y & = t^{-2}\sin{t} + 0t^{-2} \\
\Aboxed{y & = \frac{\sin{t}}{t^2}}
\end{align*}

\end{solution}

\pagebreak

\problem [21] Consider the initial value problem 
$$y^{\prime}-\frac{3}{2}y=3t+2e^{t};\: y(0)=y_0$$ Find the value of $y_0$ that 
separates solutions that grow positively as $t \to \infty$ from those 
that grow negatively. How does the solution that corresponds to this critical 
value of $y_0$ behave as $t \to \infty$?

\begin{solution}

\step \begin{align*}
P(t) & = -\frac{3}{2} \\
G(t) & = 3t + 2e^t 
\end{align*}

\step Now find the integrating factor $u(t)$:

\step \begin{align*}
u(t) & = e^{\int P(t) dt} \\
u(t) & = e^{- \int \frac{3}{2} dt} \\
u(t) & = e^{- \frac{3t}{2}}
\end{align*}

\step Now use the integrating factor:

\step \begin{align*}
\frac{1}{e^{\frac{3t}{2}}} \cdot \left(y^{\prime}-\frac{3}{2}y\right) & = 
\left(3t+2e^{t}\right) \cdot \frac{1}{e^{\frac{3t}{2}}}\\
\frac{y^{\prime}}{e^{1.5t}} - \frac{3y}{2e^{1.5t}} & = \frac{3t}{e^{1.5t}} 
+ \frac{2e^{t}}{e^{1.5t}} \\
\frac{y^{\prime}}{e^{1.5t}} - \frac{3y}{2e^{1.5t}} & = \frac{3t}{e^{1.5t}} 
+ \frac{2}{e^{0.5t}} \\
\left(e^{-1.5t}y\right)^{\prime} & = 3te^{-1.5t} + 2e^{-0.5t} \\
\int \left(e^{-1.5t}y\right)^{\prime} dt & = \int \left(3te^{-1.5t} 
+ 2e^{-0.5t}\right) dt \\
e^{-1.5t}y & = - \frac{2}{3}e^{-1.5t} \left(3t+6e^t+2\right) + c \\
y & = - \frac{2}{3} \left(3t+6e^t+2\right) + ce^{1.5t} \\
y & = -2t - 4e^t - \frac{4}{3} + ce^{1.5t}
\end{align*}

\step Now use the initial value:

\step \begin{align*}
y_0 & = -2(0) - 4e^0 - \frac{4}{3} + ce^0 \\
y_0 & = c - \frac{16}{3} \\
c & = y_0 + \frac{16}{3} 
\end{align*}

\step Substitute the new value for $c$:

\step \begin{align*}
y & = -2t - 4e^t - \frac{4}{3} + ce^{1.5t} \\
y & = -2t - 4e^t - \frac{4}{3} + y_0e^{1.5t} + \frac{16}{3}e^{1.5t}
\end{align*}

\step Now let us analyze the solution to find values of $y_0$ that grow 
positively as $t \to \infty$:

\step \begin{align*}
\lim_{t \to \infty} \left(-2t - 4e^t - \frac{4}{3} + y_0e^{1.5t} 
+ \frac{16}{3}e^{1.5t}\right) & = \infty \\
\lim_{t \to \infty} \left(-2t - 4e^t + y_0e^{1.5t} 
+ \frac{16}{3}e^{1.5t}\right) & = \infty
\end{align*}

\step We can see that the terms containing $e^{1.5t}$ will grow faster than the 
linear and exponential rate of the other two terms. Thus, they will only 
determine the value of the limit when $y_0 = -\frac{16}{3}$, in which case the 
limit will approach $-\infty$. If $y_0 > - \frac{16}{3}$, the value of limit 
as $t \to \infty$ will instead be $\infty$:

\step \begin{gather*}
\boxed{y_0 = - \frac{16}{3}} \\
\boxed{y \to - \infty \: \text{as} \: t \to \infty \: \text{for} \: y_0 
= -\frac{16}{3}}
\end{gather*}

\end{solution}

\end{problems}

\mysection{2}{Separable Differential Equations}

\begin{instructions}
In each of Problems 1 through 8, solve the given differential equation.
\end{instructions}

\begin{problems}

\problem [3] $y^{\prime}=\cos^2{(x)}\cos^2{(2y)}$

\begin{solution}

\step \begin{align*}
y^{\prime} & = \cos^2{(x)} \cos^2{(2y)} \\
\frac{dy}{dx} & = \cos^2{(x)} \cos^2{(2y)} \\
dy & = \cos^2{(x)} \cos^2{(2y)} dx \\
\sec^{2}{(2y)} dy & = \cos^2{(x)} dx \\
\int \sec^{2}{(2y)} dy & = \int \cos^2{(x)} dx \\
\frac{1}{2} \tan{(2y)} & = \frac{1}{2} \left(x + \sin{(x)}\cos{(x)}\right) 
+ c \\
\tan{(2y)} & = x + \sin{(x)}\cos{(x)} + 2c \\
2\tan{(2y)} & = 2x + 2\sin{(x)}\cos{(x)} + 4c \\
%2y & = \arctan{(x + \sin{(x)}\cos{(x)} + 2c)} \\
%y & = \frac{1}{2} \arctan{(x + \sin{(x)}\cos{(x)} + 2c)} \\
%y & = \frac{1}{2} \arctan{(x + \sin{(x)}\cos{(x)} + c)} \\
2\tan{(2y)} & = 2x + 2\sin{(2x)} + 4c 
\end{align*}

\step Below is a valid solution, given that the domain is restricted such that 
$\cos{(2y)} \neq 0$.

\step \begin{align*}
\Aboxed{2\tan{(2y)} & = 2x + 2\sin{(2x)} + c}
\end{align*}

\end{solution}

\pagebreak

\problem [5] $\frac{dy}{dx}=\frac{x-e^{-x}}{y+e^y}$

\begin{solution}

\step \begin{align*}
\frac{dy}{dx} & = \frac{x-e^{-x}}{y+e^y} \\
\left(y+e^y\right)dy & = \left(x-e^{-x}\right) dx \\
\int \left(y+e^y\right)dy & = \int \left(x-e^{-x}\right) dx \\
\frac{y^2}{2} + e^y & = \frac{x^2}{2} + e^{-x} + c
\end{align*}

\step Below is a valid solution, given that the domain is restricted such that
$y+e^y \neq 0$:

\step \begin{align*}
\Aboxed{y^2 + 2e^y & = x^2 + 2e^{-x} + c}
\end{align*}

\end{solution}

\end{problems}

\begin{instructions}
In each of Problems 9 through 16:
\begin{subinstructions}
\subinstruction Find the solution of the given initial value problem in 
explicit form.
\subinstruction \sout{Plot the graph of the solution.}
\subinstruction Determine (at least approximately) the interval in which the
solution is defined.
\end{subinstructions}

\end{instructions}

\begin{problems}

\problem [9] $y^{\prime} = (1-2x)y^2$, $y(0) = -1/6$

\begin{solution}

\step \begin{align*}
\frac{dy}{dx} & = (1-2x)y^2 \\
\frac{dy}{y^2} & = (1-2x)dx \\
\int \frac{dy}{y^2} & = \int (1-2x)dx \\
-\frac{1}{y} & = - x^2 + x + c
\end{align*}

\step Now use the initial value to find $c$:

\step \begin{align*}
6 & = 0 + 0 + c \\
c & = 6
\end{align*}

\step Now substitute for c:

\step \begin{align*}
-\frac{1}{y} & = - x^2 + x + 6 \\
\Aboxed{y & = \frac{1}{x^2 - x - 6}}
\end{align*}

\step Analyzing the solution to the differential, we can see that a restriction 
must be placed on the domain such that $x^2 - x - 6 \neq 0$. 

\step \begin{align*}
x^2 - x - 6 & = 0 \\
(x-3)(x+2) & = 0 \\
x & = \{-2,3\} \\
\end{align*}

\step Thus, the domain for x is restricted by the solution such that $x \neq 
\{-2, 3\}$. Because of this, and that the initial value given was at $x=0$, we 
can guarantee this solution over the following domain:

\step \begin{gather*}
\boxed{\{x \in \R \ssep -2 < x < 3\}} 
\end{gather*}

\end{solution}

\problem [10] $y^{\prime} = (1-2x)/y$, $y(1) = -2$

\begin{solution}

\step \begin{align*}
\frac{dy}{dx} & = (1-2x)/y \\
y dy &= (1-2x)dx \\
\int y dy & = \int (1-2x)dx \\
\frac{y^2}{2} & = x - x^2 + c
% y^2 & = 2x - 2x^2 + 2c \\
\end{align*}

\step Now find the value of c using the initial value:

\step \begin{align*}
\frac{(-2)^2}{2} & = 1 - 1^2 + c \\
2 & = 1 - 1 + c \\
c & = 2 
\end{align*}

\step Now substitute c and simplify:

\step \begin{align*}
\frac{y^2}{2} & = x - x^2 + 2 \\
y^2 & = 2x - 2x^2 + 4 \\
\Aboxed{y & = \pm \sqrt{2x - 2x^2 + 4}}
\end{align*}

\step Looking at this solution, we can see that restrictions must be placed
on the domain if the range is to be restricted to $\R$. That is, $2x - 2x^2
+ 4 > 0$:

\step \begin{gather*}
2x - 2x^2 + 4 = 0 \\
2x - 2x^2 + 4 = 0 \\
-2(x^2-x-2) = 0 \\
-2(x-2)(x+1) = 0 \\
\boxed{\{x \in \R \ssep -1 < x < 2\}}
\end{gather*}

\end{solution}

\end{problems}

\begin{instructions}
Some of the results requested in Problems 17 through 22 can be
obtained either by solving the given equations analytically or by
plotting numerically generated approximations to the solutions. Try
to form an opinion about the advantages and disadvantages of each
approach.
\end{instructions}

\begin{problems}

\problem [17] Solve the initial value problem $$y^{\prime}=\frac{1+3x^2}
{3y^2-6y},\:y(0)=1$$ and determine the interval in which the solution is valid.

\begin{solution}

\step \begin{align*}
\frac{dy}{dx} & = \frac{1+3x^2}{3y^2-6y} \\
\int (3y^2 - 6y) dy & = \int (1+3x^2) dx \\
y^3 - 3y^2 & = x + x^3 + c
\end{align*}

\step Now use the initial value to find c:

\step \begin{align*}
1^3 - 3(1^2) & = 0 + 0^3 + c \\
1 - 3 & = c \\
c & = -2
\end{align*}

\step Now substitute for c:

\step \begin{align*}
\Aboxed{y^3 - 3y^2 & = x^3 + x - 2}
\end{align*}

\step Analyzing the original differential equation, we can see that domain must 
be restricted such that $3y^2 - 6y \neq 0$:

\step \begin{align*}
3y^2 - 6y = 0 \\
y^2 - 2y = 0 \\
y^2 - 2y + 3 = 3 \\
(y-3)(y+1) = 3 \\
y = \{0,2\}
\end{align*}

\step \begin{align*}
0^3 - 3(0^2) & = x^3 + x - 2 \\
0 & = x^3 + x - 2 \\
0 & = (x-1)(x^2 + x + 2) \\
x & = 1
\end{align*}

\step \begin{align*}
2^3 - 3(2^2) & = x^3 + x - 2 \\
0 & = x^3 + x + 2 \\
0 & = (x+1)(x^2 - x + 2) \\
x & = -1 
\end{align*}

\step Because the initial value is at $x=0$ and the values where the 
differential does not exist, the solution can be guaranteed over the following 
domain:

\step \begin{gather*}
\boxed{\{x \in \R \ssep -1 > x > 1\}}
\end{gather*}

\end{solution}

\end{problems}

\begin{instructions}
The method outlined in Problem 25 can be used for any homogeneous equation.
That is, the substitution $y = xv(x)$ transforms a homogeneous equation into a 
separable equation. The latter equation can be solved by direct integration, 
and then replacing $v$ by $y/x$ gives the solution to the original equation. In 
each of Problems 26 through 31:

\begin{subinstructions}
\subinstruction Show that the given equation is homogeneous.
\subinstruction Solve the differential equation.
\subinstruction \sout{Draw a direction field and some integral curves. Are they 
symmetric with respect to the origin?}
\end{subinstructions}

\end{instructions}

\begin{problems}

\problem [27] $\frac{dy}{dx}=\frac{x^2+3y^2}{2xy}$

\begin{solution}

\step 

\end{solution}

\problem [28] $\frac{4y-3x}{2x-y}$
\end{problems}

\mysection{3}{Modeling with First-Order Differential Equations}

\begin{problems}
\problem [1] Consider a tank used in certain hydrodynamic experiments.
After one experiment the tank contains $200\:L$ of a dye solution with
a concentration of $1\:g/L$. To prepare for the next experiment, the tank
is to be rinsed with fresh water flowing in at a rate of $2\:L/min$, the
well-stirred solution flowing out at the same rate. Find the time that
will elapse before the concentration of dye in the tank reaches $1\%$ of
its original value.
\problem [12] Newton’s law of cooling states that the temperature of an object
changes at a rate proportional to the difference between its temperature
and that of its surroundings. Suppose that the temperature of a cup of
coffee obeys Newton’s law of cooling. If the coffee has a temperature
of $200^\circ\:F$ when freshly poured, and $1\:min$ later has cooled to 
$190^\circ\:F$ in a room at $70^\circ\:F$, determine when the coffee reaches a 
temperature of $150^\circ\:F$.
\problem [16] A ball with mass $0.15\:kg$ is thrown upward with initial
velocity $20\:m/s$ from the roof of a building $30\:m$ high. Neglect air
resistance.

\begin{subproblems}
\subproblem Find the maximum height above the ground that the ball
reaches.
\end{subproblems}

\end{problems}

\end{document}